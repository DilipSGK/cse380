\documentclass[letterpaper,12pt]{article}
\setlength{\topmargin}{0in}
\setlength{\textheight}{9.0in}
\setlength{\textwidth}{6.5in}
\setlength{\columnsep}{0.25in}
\setlength{\headheight}{0in}
\setlength{\headsep}{0in}
\setlength{\oddsidemargin}{0.0in}
\setlength{\evensidemargin}{0.0in}
\setlength{\topskip}{0in}
\usepackage{times}
\usepackage{mdwlist}
\usepackage{hyperref}
\usepackage{array}
\usepackage{multicol}
\usepackage{multirow}
\usepackage{enumitem}
\usepackage{graphicx}
\usepackage{amsmath}
\usepackage{parskip}
\hypersetup{colorlinks=true}
\renewcommand{\baselinestretch}{1.00}
\DeclareMathOperator{\erf}{erf}

\newcounter{exanum}
\setcounter{exanum}{1}
\newcommand{\exercise}{\subsubsection*{Exercise \#\theexanum: \refstepcounter{exanum}}}

\newcommand{\duedate}{end of day, October 31, 2019}

\input{vc}
\newcommand{\courseyear}{Fall 2019}
\newcommand{\system}{Stampede2}


\begin{document}
\begin{flushright}
\VCDateISO \\
Rev: \VCRevision
\end{flushright}

\vspace*{0.05cm}
\subsection*{Tools and Techniques for Computational Science - \courseyear{}}
\vspace*{-.2cm}
{{\large \em Assignment \#4}

\vspace*{5pt}

\subsection*{Modeling Document}

This homework is designed to help you prepare for completing your
first project, which involves writing an application to solve the steady-state heat
equation in one- and two-dimensions. To prepare for this endeavor, you will
begin by drafting your {\em ``model document''} which highlights the
governing equations, boundary conditions, numerical approximations,
and high-level pseudo-code you will use to implement your
solver. Note that this model document will then later expand as you write your
application to include a verification methodology, input/runtime
options, build procedures, and example results.

Commit a single PDF and the associated \LaTeX files necessary to build your
modeling document into 
your assigned GitHub repository
(https://github.com/uthpc/student-{\em yourgithubId}.git).  Please
store the results in a \texttt{hw04/} subdirectory and tag your repo with a
name of ``hw04'' after completing the assignment. This homework is due by \duedate. \\

\rule{\textwidth}{0.4pt} 

\exercise

(100 pts) The steady-state heat equation with a constant coefficient in two dimensions is 
given by:
\begin{equation}
-k\nabla^{2} T(x,y) = q(x,y)
\end{equation}

where $k$ is the thermal conductivity, $T$ is the material
temperature, and $q$ is a heat source term.  Ultimately, we wish to
solve this numerically via multiple finite-difference stencils and
linear system solution techniques, and verify our results. In
preparation for this, create a modeling document in your git repo
using \LaTeX which describes all the pertinent details needed to
implement your solver. The minimum items to include in your modeling
document for this assignment include:

\begin{itemize*}
\item Governing equations, nomenclature, and boundary conditions.  You
  can assume Dirichlet boundary conditions for this effort. 
\item List of any assumptions you are making.
\item Numerical Method:
  \begin{itemize*}
  \item provide 2nd- and 4th-order (central) finite-difference
    approximations for the second derivative in the heat equation. Make
    sure to include the leading order of the truncation error.
    Provide discrete approximations of the heat equation using these
    formulations (again including truncation error).
\item provide representative figures of 1D and 2D discretized meshes
  indicating domain size, mesh sizes (including variable names that will be
  referenced in your input file later on), and finite-difference notation
  for neighboring cells.  Be sure to indicate whether your scheme is
  node-based or cell-based. You can assume a square domain for the 2D
  case (but make sure to include this in your list of assumptions).
  \item outline the linear system that you will form as part of the
    numerical method. Indicate the number of non-zero entries on an
    interior row of the matrix for the 1D, and 2D cases with both
    finite-difference schemes (bonus points will be given for nice
    figures which highlight the structure of the resulting matrices).
   \item in pseudo-code, provide a simple iterative solution mechanism for the
     resulting linear system using:
     \begin{enumerate*}
       \item Jacobi
       \item Gauss-Seidel
       \end{enumerate*}
\item estimate the amount of memory required (using double-precision)
  to implement your numerical methods
\end{itemize*}
\end{itemize*}



\end{document}
