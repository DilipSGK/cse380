\documentclass[letterpaper,12pt]{article}

\setlength{\topmargin}{0in}
\setlength{\textheight}{9.0in}
\setlength{\textwidth}{6.5in}
\setlength{\columnsep}{0.25in}
\setlength{\headheight}{0in}
\setlength{\headsep}{0in}
\setlength{\oddsidemargin}{0.0in}
\setlength{\evensidemargin}{0.0in}
\setlength{\topskip}{0in}
\usepackage{times}
\usepackage{mdwlist}
\usepackage{hyperref}
\usepackage{array}
\usepackage{multicol}
\usepackage{enumitem}
\hypersetup{colorlinks=true}
\renewcommand{\baselinestretch}{1.00}
\newcommand{\courseyear}{Fall 2021}
\newcommand{\courseyearonly}{2021}
\newcommand{\system}{Stampede2}

\newcounter{exanum}
\setcounter{exanum}{1}
\newcommand{\exercise}{\subsubsection*{Exercise \#\theexanum: \refstepcounter{exanum}}}

\newcommand{\duedate}{end of day, \underline{September 21, 2021}}

\begin{document}
\begin{flushright}
\VCDateISO \\
Rev: \VCRevision
\end{flushright}

\vspace*{0.1cm}
\subsection*{Tools and Techniques for Computational Science - \courseyear{}}
\vspace*{-.2cm}
\noindent{{\large \em Assignment \#1}

\vspace*{10pt}

\subsection*{Architecture, System Access, and Starting Linux}

Please answer the following questions and upload a single PDF with your responses to the
class Canvas site.  This homework is due by \duedate.  

\exercise

(20 pts) Consult the \href{https://www.top500.org/system/179045}{Top500} entry
for the \system{} system at TACC and the system user
guide. Calculate the theoretical peak (double-precision floating point) performance 
of the system and show exactly how this number is calculated. \\ 

\vspace*{3cm} % room for answer

\noindent Is your answer the same or different than what is reported
for \texttt{Rpeak}? If different, why might this be the case?

\vspace*{2cm} % room for answer

\exercise

(15 pts) Prove that you can access the \system{} system using ssh. Once logged
in, your mission is to find a ``secret key'' that is located somewhere
in karl's \texttt{WORK} directory (/work/00161/karl/stampede2). Hint:
the file you are seeking is named \texttt{WumpuS} and it contains
multiple records (one per student). Find the entry matching your TACC
User ID and update the table below.

\vspace*{0.5cm}

\begin{table}[h]
\begin{tabular}{c|>{\centering\arraybackslash}p{8cm}}
\texttt{TACC User Name} & \texttt{SECRET KEY} \\ \hline
 &  \\ 
 &  \\ \hline
\end{tabular}
\end{table}

\newpage
\subsection*{Filters and Pipelines}

A \textit{filter} is a small and (usually) specialized program in Unix-like
operating systems that transforms data in some way.  As is typically the case
with command line programs in Unix, filters read data from standard input and
write to standard output.  Standard input is the data source for the program;
by default, this is the input from the keyboard. However, programs (and thus
filters) can receive input from a file or from the output of another
program. This is referred to as \textit{redirection}.

Standard output is the display by default. Thus, if standard output
for a program is not redirected to a file or a device (or another
program), it will be displayed directly to your terminal. If, however,
you want to use output from one program as input for another program
(a filter, for instance), you can connect the data streams with
\textit{pipes}.

\smallskip

\noindent \textbf{By using filters, redirection, and pipes, it is possible to
  construct a sequence of commands to accomplish a highly specialized
  task.}

\bigskip

\noindent \textbf{Example:} How many users on \system{} have a name that contains the string ``Chris''?

\begin{verbatim}
$ getent passwd | grep Chris | wc -l
305
\end{verbatim}

Note that we often refer to command strings like the example above as
{\em one-liners} in that they accomplish a desired function with a
single line of Unix commands.   \\

\noindent Examples of other common filters include:

\begin{multicols}{3}
\begin{itemize}[nolistsep]
    \item awk
    \item cat
    \item cut
    \item expand
    \item fold
    \item grep
    \item head
    \item more (less)
    \item sed 
    \item sort
    \item strings
    \item tail
    \item tac
    \item tee
    \item uniq
    \item wc
\end{itemize}
\end{multicols}

\noindent Leveraging your fine Linux skills, answer the following ({\em show the
  commands used in your responses}):

\exercise
(30 pts) How many users on \system{} use bash? How many use tcsh? What is the most
popular shell that isn't bash or tcsh?

\vspace*{4cm}

\exercise

(15 pts) Who has the longest TACC username? How many characters is it? How many TACC
users have usernames that are exactly 8 characters long?


\vspace*{4cm}

\exercise

(20 pts) Many Linux distributions make a dictionary file of English words available.
Some programs use this database of words for spell-checking purposes. Password
checkers use it to look for passwords that can be easily cracked.  A copy of
this file from the CentOS distribution is available in karl's \texttt{WORK}
directory named: \texttt{linux.words} Examine this file and describe how the
entries are arranged and formatted. Grab a copy of this file to keep in your
own directory to complete this exercise.

\medskip

\noindent Construct one liners using regular expressions to answer the following questions:
\begin{enumerate}
\item count the number of words beginning with each letter of the alphabet and prints the result in the following format:
\begin{verbatim}
a:31788
b:25192
...
\end{verbatim}

\item How many words end with the letter ``s''? How how entries contain a possessive word (e.g. owl's-crown)?

\vspace*{1.5cm}

\item Write a one-liner that counts and prints out the number of words that contain the substring \textit{foo} but not the 
substrings \textit{food, fool} or \textit{foot}. (Hint: this can be done most logically with 1, 2, or 4 calls to grep. 
See if you can do it in 1, but full credit will be given for any answer that works.) 

\vspace*{1.5cm}

\item Generate your own question involving parsing the dictionary file
and then generate the one-liner necessary to answer your question.
\end{enumerate}

\end{document}

